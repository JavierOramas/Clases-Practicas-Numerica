\documentclass{article}

    \usepackage[margin=0.7in]{geometry}
    \usepackage[parfill]{parskip}
    \usepackage[utf8]{inputenc}
    

    \usepackage{amsmath,amssymb,amsfonts,amsthm}

\begin{document}
\section{ Ejercicio 1: En este ejercicio hay varios errores (10000 créditos)}
Alguien calculó 1.41, 0.02573 y 846.28 como aproximaciones de $\sqrt{2}$, 0.024 y 847. 
\end{itemize} 
\subsubsection{ Respuesta:}
$$ |e| = |x - \bar{x}| $$  $$ |e_{1}| = |1.41421356237 - 1.41| = 0.00421356237 $$ $$ |e_{2}| = |0.024 - 0.02573| = 0.00173 $$ $$ |e_{3}| = |847 - 846.28| = 0.72 $$


\end{itemize} 
\subsubsection{ Respuesta:}
$$ \delta = \frac{|e|}{|x|} $$ $$ \delta_{1} = \frac{|e_{1}|}{|\sqrt{2}} =  0.0029794385 $$ $$ \delta_{2} = \frac{|e_{2}|}{|0.024|} = 0.720833333 $$ $$ \delta_{3} = \frac{|e_{3}|}{|847|} = 0.000850059 $$


\end{itemize} 
\subsubsection{ Respuesta:}
$$ |\delta| \leq  \frac{1}{2}\beta^{1-k} $$ 
\end{itemize} 	$$ 0.0029794385 \leq 0.005 $$ (tiene 3 cifras significativas correctas) $$ |\delta_{1}| \leq \frac{1}{2}10^{1-4} $$ $$ 0.0029794385 \not\leq 0.0005 $$ (no tiene 4 cifras significativas correctas) 
\end{itemize} 	$$ 0.720833333 \leq 5 $$ (tiene 0 cifras significativas correctas) 	$$ |\delta_{2}| \leq \frac{1}{2}10^{1-1} $$ $$ 0.0.720833333 \not\leq 0.5 $$ (no tiene 1 cifra significativa correcta) 
\end{itemize} 	$$ 0.000850059 \leq 0.005 $$ (tiene 3 cifras significativas correctas) $$ |\delta_{3}| \leq \frac{1}{2}10^{1-4} $$ $$ 0.000850059 \not\leq 0.0005 $$ (no tiene 4 cifras significativas correctas)

d) ¿Cuál es la aproximación más precisa? 
\subsubsection{ Respuesta:}
La mejor aproximación es la de $ 847 \approx 846.28 $ dado que el error relativo es el menor
\section{ Ejercicio 2: e de error (15000 créditos)}
El número e se puede calcular como $ e = \sum^{\infty}_{i=0} \frac{1}{n!}$. Calcule el error absoluto y relativo que se obtienen al realizar las siguientes aproximaciones: a) $ e \approx \sum_{i=0}^{5}\frac{1}{n!}$ 
\subsubsection{ Respuesta:}
$$ \sum_{i=0}^{5}\frac{1}{n!} = 2.708333333333333 $$ $$ |e_{1}| = |2.7182818284590445 - 2.708333333333333| = 0.009948495125712054 $$ $$ \delta = \frac{|e_{1}|}{|e|} =  0.003659846827343768 $$ b) $ e \approx \sum_{i=0}^{5}\frac{1}{n!}$ 
\subsubsection{ Respuesta:}
$$ \sum_{i=0}^{5}\frac{1}{n!} = 2.7182815255731922 $$ $$ |e_{2}| = |2.7182818284590445 - 2.7182815255731922 | = 3.0288585284310443e-07 $$ $$ \delta = \frac{|e_{2}|}{|e|} = 1.1142547828265698e-07 $$
\section{ Ejercicio 3: Este ejercicio se puede hacer con una condición (20000 créditos)}
Calcule cond(f), para la función $ f(x) = (x − 1)^2$.

$$ cond(f) = \frac{|x * 2(x-1)|}{(x-1)^{2}} = \frac{|2x|}{x-1} = \frac{|2|}{|1-\frac{1}{x}|}$$

a) Determine un valor de x para el que f(x) esté bien condicionada.


\subsubsection{ Respuesta:}
$$ -1 < \frac{2x}{x-1} < 1 $$ $$ -(x-1) < 2x < x-1 $$ $$ -(x-1) < 2x $$ $$ 1 < 3x $$ $$ x > \frac{1}{3} $$

$$ 2x < x-1 $$ $$ x < -1 $$

Está bien condicionada $ \forall x : x \in (-\infty,-1) \cup (\frac{1}{3}, +\infty)$

b) ¿Para qué valores de x la función está mal condicionada? 
\subsubsection{ Respuesta:}
Está bien condicionada $ \forall x : x \in [-1,\frac{1}{3}]$
\section{ Ejercicio 4: Diga NO a la inestabilidad de calculo (40000 créditos)}
Dadas las funciones $ f(x) =\sqrt{x + 4} − 2, g(x) = 1 − e^{x} $:

a) ¿Qué problema numérico aparece al evaluar f(x) y g(x) en valores de x que estén cercanos a 0? 
\subsubsection{ Respuesta}
Inestabilidad de cálculo o cancelación catastrófica por tomar valores cercanos a 0

b) Encuentre expresiones equivalentes para f(x) y g(x) en las que no aparezca el problema detectado en el inciso anterior. 
\subsubsection{ Respuesta:}
$$ f(x) = \sqrt{x+4} - 2 \frac{\sqrt{x+4}+2}{\sqrt{x+4}+2} = \frac{x+4-4}{\sqrt{x+4}+2} = \frac{x}{\sqrt{x+4}+2}$$

Utilizando Serie de Taylor:

$$ g(x) = 1-e^{x} \approx 1 - (1 + x + \frac{x^{2}}{2} + \frac{x^{3}}{6}) \approx 1 - 1 -(x + \frac{x^{2}}{2} + \frac{x^{3}}{6}) \approx -(x + \frac{x^{2}}{2} + \frac{x^{3}}{6}) $$
\section{ Ejercicio 5: Si no se conoce el valor real... ¡No importa! (40000 créditos)}
Cuando se trabaja con métodos numéricos nunca se conoce el valor real que se esté buscando. ¡Eso hace que en la práctica sea imposible calcular el error relativo :-o! Sin embargo, en la párctica se suele usar el valor $\bar{\delta} $ que se calcula como:

$$ \bar{\delta} =\frac{|x − \bar{x}|}{|\bar{x}|} $$

Lo que hay que hacer en este ejercicio es que si δ ≈ 0, entonces $ \bar{\delta}$


\end{itemize}


\subsubsection{ Respuesta:}
$$ \frac{|x-\bar{x}|}{|x|} \approx 0 $$ $$ |x-\bar{x}| \approx 0 $$ pero si $ |x-\bar{x}| \approx 0 $, entonces: $$ \frac{|x-\bar{x}|}{|\bar{x}|} \approx 0 $$ luego: $$ \frac{|x-\bar{x}|}{|x|} \approx \frac{|x-\bar{x}|}{|\bar{x}|} $$ $$ \delta \approx \bar{\delta} $$
\section{ Ejercicio 6: ¿Puedes hacer el ejercicio en abstracto? :-/ (40000 créditos)}
Sean $ h(x) = \sqrt{x + 1} − \sqrt{x} $ y F = (β, p, m, M) una aritmética de punto flotante.


\end{itemize}

a) ¿Para qué valor de x, hay “problemas” al evaluar h(x) en la aritmética F. 
\subsubsection{ Respuesta:}
$$ x < \beta^{m-p} $$

b) Cómo se puede evaluar h(x) evitando el problema indicado en el inciso anterior. 
\subsubsection{ Respuesta:}
$$ \sqrt{x+1} - \sqrt{x} * \frac{\sqrt{x+1}+ \sqrt{x}}{\sqrt{x+1} + \sqrt{x}} = \frac{x+1-x}{\sqrt{x+1}+\sqrt{x}} = \frac{1}{\sqrt{x+1}+\sqrt{x}} \approx \frac{1}{2\sqrt{x}}$$
\section{ Ejercicio 7: Para no discriminar contenidos (50000 créditos)}
Una de las soluciones de la ecuación cuadrática $ ax^2 + bx + c = 0 $ se puede obtener como:

$$ x_1 =\frac{−b +\sqrt{b^2 − 4ac}}{2a} $$

a) Para qué valores de a, b y c esta forma de calcular x1 resulta numéricamente inestable en una arimética F(β, p, m, M). 
\subsubsection{ Respuesta:}
$$ 4ac = 0$$ $$ a = 0 \lor c = 0 $$

b) Utilizando los conceptos de estabilidad de cálculo y condición de una función, justifique por qué los valores determinados en el inciso anterior para a, b y c hacen que esa forma de calcular x1 sea inestable. 
\subsubsection{ Respuesta:}

\section{ Ejercicio 8: Para vincular conceptos e ideas (60000 créditos)}
Explique el fenómeno de cancelación catastrófica en términos de la condición de la resta. Para ello: a) Argumente qué tiene que ver la cantidad de cifras significativas correctas con el error relativo. 
\subsubsection{ Respuesta:}
Se enuentran relacionadas por la fórmula: $$ |\delta| = \frac{1}{2}\beta^{1-k} $$ (con $\delta$ como error relativo y k la cantidad de cifras significativas correctas)

b) Explique qué tiene que ver el error relativo con la condición de una función. 
\subsubsection{ Respuesta:}
$$ cond(f) = \frac{\frac{|f(\bar{x})-f(x)|}{|f(x)|}}{\frac{|\bar{x}-x|}{|x|}} = \frac{|\delta|}{\frac{e}{|x|}} = \frac{e|\delta|}{|x|} $$

c) Determine para qué valores de x la función f(x) = x − a, con a ∈ R está mal condicionada. 
\subsubsection{ Respuesta:}
$$ cond(f) = \frac{xf^{'}(x)}{f(x)} = \frac{x}{x-a} \implies \lim_{x\to a}\frac{x}{x-a} = \infty $$ cunado x = a la funciomn está mal condicionada

d) Usa los resultados de los incisos anteriores e impresiona a los profesores ;-).
\section{ Ejercicio 9: El problema con las cifras significativas correctas (Al menos 30000 créditos)}
Es un hecho comúnmente aceptado que tus resultados matemáticos serán tan buenos como las definiciones de las que partas. La idea de cifras significativas correctas tiene el inconveniente de que para casi cualquier definición “medianamente sensata e intuitiva” que se dé, se pueden encontrar ejemplos que no “encajan” con esa definición. Ponga al menos dos ejemplos de estas posibles definiciones “sensatas” y cuáles son los problemas que tienen.


\subsubsection{ Respuesta:}

\end{itemize}


\end{itemize}


\end{itemize}


\end{itemize}
\section{ Ejercicio 10: Tipos de errores (60000 créditos)}
Por la forma en que se calculan, los errores pueden ser absolutos o relativos, pero de acuerdo a su origen se pueden clasificar en tres grandes grupos: de redondeo, de truncamiento e inherentes. a) Diga en qué consiste cada uno de estos “errores”. 
\end{itemize}


\end{itemize}


\end{itemize}

b) Pon al menos dos ejemplos de cada uno de ellos. 
\end{itemize}
\section{ Ejercicio 11: Errores catastróficos (de verdad) (Al menos 30000 créditos)}
Para ponerle emoción a la asignatura, en este ejercicio hay explosiones, muertes, y misterios de la vida real, todos relacionados con tus recientes conocimientos de Matemática Numérica :-o.

a) Busca ejemplos de catástrofes y problemas reales que hayan ocurrido por culpa del mal uso de la matemática numérica.

b) Describe y explica (si puedes) los elementos de matemática numérica que intervinieron en esas catástrofes.

c) ¿Puedes decir cómo se hubieran podido evitar esos problemas?


\end{itemize}


\end{itemize}


\end{itemize}
\section{ Ejercicio 13: Precisando detalles (200000 créditos)}
¿Dada una aritmética F = (β, p, m, M), y una función f : R → R, para qué valores de cond(f), la función está mal condicionada?


\subsubsection{ Respuesta:}
Está mal condicionada cuando $ cond(f) = |\frac{\bar{f(x)}}{f(x)}x| > 1$
\end{document}
